\documentclass[a4paper,12pt]{article}
\usepackage[utf8]{inputenc}
\usepackage[russian]{babel}
\usepackage[left=2cm,right=2cm,
    top=1cm,bottom=2cm,bindingoffset=0cm]{geometry}
    \usepackage[dvips]{graphicx}
    \usepackage{graphicx}%пакет для вставки изображений
    %\graphicspath{{piсs/}}%указываем откуда следует вставлять изображения
\DeclareGraphicsExtensions{.png,.jpg,.svg}%какие расширение будем использовать 
%\usepackage{fullpage}
\usepackage{subcaption}
\usepackage[dvinames, svgnames, x11names]{xcolor}%расширенная палитра
\usepackage{pgfplots}
\usepackage{tikz}
\usepackage[unicode, pdftex]{hyperref}
\usepackage{color,soul}
\begin{document}


Для работы с графикой мы решили рассмотреть два пакета: \{graphicx\} и \{tikz\} первый служит для вставки растровых изображений в текст,  а второй позволяет выполнять построение различных геометрических фигур, блок-схем, а тыкже графиков некоторых функций, что представляет гораздо больший интерес. Начнем с пакета \{graphicx\}. \\
Для нам нужно подключить его в преамбуле документа:\\
\\
\hypertarget{D12}
$\backslash$\textbf{usepackage \{graphicx \}} \\
\textbf{\graphicspath\{\{pictures/ \}\} } Указываем название каталога где будут лежать изображения.(Он должен находиться в том же каталоге что и сам документ) Данная опция является необязательной, можно просто разместить все изображения в том же каталоге что и документ.\\
$\backslash$ \textbf{DeclareGraphicsExtensions\{.pdf,.png,.jpg\}} Указываем какие типы файлов будем использовать.Векторные изображения также поддерживаются. \\
Рассмотрим вставку изображений:\\
$\backslash$ \textbf{begin \{figure\}[h!]} "Обьявляем начало" изображения, в квадратных скобках указываем позицию изображения, "h!" обозначает, что изображение будет вставлено сразу после текста.       \\
$\backslash$ \textbf{ setlength \{ $\backslash$ fboxsep \} \{0pt \} } размер полей вокруг изображения     \\
$\backslash$ \textbf{setlength \{ $\backslash$ fboxrule \} \{ 1pt \} } ширина рамки \\
$\backslash$ \textbf{fbox \{ $\backslash$ includegraphics [ width=15cm,height=9cm ]\{ Matrrix \( 1 \) \} \} } задаем размеры изображения и указываем название файла(файл должен лежать в одной папке с документом) 
 
Примеры вставки изображений есть в основном отчете.
 
$\backslash$ \textbf{caption \{ "Крупные' математические объекты \}}| Подпись под изображением  \\
$\backslash$ \textbf{end \{ figure \}  } "Конец" \\
Если подпись и рамка не трубуются, то достачно только строчки $\backslash$ \textbf{fbox \{ $\backslash$ includegraphics [ width=15cm,height=9cm ]\{ Matrrix \( 1 \) \} \} }. Вместо усазания размеров в сантиметрах можно использовать команду \textbf{scale} \( масштаб\). \\
Итак, перейдем к пакету \textbf{\{tikz\}}. Его базовая функция - начертание фигур по их координатам.\\

\begin{tikzpicture}
\draw(0,0)--(1,0)--(1.5,-0.5)--(1.5,-2)--(1,-3.5)--(2.5,-4.5)--(3,-4.5)--(3,-4)--(2.5,-4)--(2,-3)--(3,-2.5)--(3.5,-3.5)--(4.5,-4)--(5,-4)--(5,-3.5)--(4.5,-3.5)--(4,-3)--(4,-2.5)--(4.5,-3)--(6.5,-3)--(7,-4.5)--(7.5,-5)--(8,-5)--(8,-4.5)--(7.5,-4.5)--(7.5,-3.5)--(8.5,-4.5)--(9,-4.5)--(9,-4)--(8.5,-4)--(8,-3)--(8.5,-3)--(9,-2.5)--(9,-1)--(8.5,-1.5)--(8,-1.5)--(7.5,-1)--(7.5,-2)--(7,-2)--(5,-1)--(4,-1)--(2.5,-1.5)--(2.5,-0.5)--(1.5,0.5)--(1,0.5)--cycle;
\end{tikzpicture}

\begin{tikzpicture}
\draw(0,0)--(1,0.5)--(2.5,0.5)--(3,0)--(2,-0.5)--(0.5,-0.5)--(0,0)--(0,-2.5)--(0.5,-3)--(0.5,-0.5)--(0.5,-3)--(2,-3)--(2,-0.5)--(2,-3)--(3,-2.5)--(3,0);
\draw[dashed](0,-2.5)--(1,-2)--(2.5,-2)--(3,-2.5);
\draw[dashed](1,-2)--(1,0.5);
\draw[dashed](2.5,-2)--(2.5,0.5);
\end{tikzpicture}


Все что нужно, это указывать координаты.При необходимости построить сложную фигуру можно строить несколько линий. Имеется возможность строить пунктирные линии, стрелки,также можно окрашивать в различные цвета,регулировать толщину.

\begin{tikzpicture}[>=stealth]
\draw[thick, ->,red](0,0)--(4,0);
\draw[thick, ->, red](0,0)--(1,3);
\draw[ultra thick, ->, blue](0,0)--(5,3);
\draw[dashed](4,0)--(5,3)--(1,3);
\end{tikzpicture}

\vspace{3cm}
\textbf{Добавление подписей к прямым и углам}\\

\begin{tikzpicture}[scale=0.75]
\fill[left color=magenta, right color=yellow]
(0,0) -- node[below=3pt] {$a$} (4,0) --
node[right=5pt] {$b$} (4,3) --
cycle node[midway,above,sloped] {$c=\sqrt{a^2+b^2}$};
\node[below left] at (0,0) {\color{blue}$B$};
\node[below right] at (4,0) {\color{blue}$C$};
\node[above right] at (4,3) {\color{blue}$A$};
\end{tikzpicture}




\textbf{Окружности и дуги}\\

\begin{tikzpicture}
\draw[Red,ultra thick](,1) arc (20:60:2);
\draw[Red,ultra thick](0,0) arc (20:60:1.5);
\draw[thick](2,1) arc (0:-120:2);
\end{tikzpicture}

В квадратных скобках указываются дополнительные параметры(как и для любых линий), затем указываются координаты центра, а затем название фигуры(в нашем случае это дуга) далее следуют градусные меры начала и конца, длина радиуса.\\

\begin{tikzpicture}
\draw [fill=Red] (0,0) circle (1);
\draw [color=green, outer color=Blue] (2,0) circle (1);
\draw [ball color=green] (1,-1.7) circle (1);
\end{tikzpicture}


В пакете \textbf{\{tikz\}} имеется огромное количество готовых фигур, однако перечислять их мы не будем, ибо на это ушло бы слишком много времени. Однако мы не можем не показать использование данного пакета для построения графиков функций.(как предустановленных, так и при помощи таблицы значений)


\begin{tikzpicture}
		\begin{axis}[
			title = Кубическая парабола,
			xlabel = {$x$},
			ylabel = {$y$},
			grid = both,
			minor tick num = 2
			]
			\addplot[magenta] {x^3};
		\end{axis}
	\end{tikzpicture}
\vspace{2cm}
\begin{tikzpicture}
\begin{axis}
grid = both,
\addplot3 table [x = b, y = a, z = c] {
	a      b      c
	1      1      1
	7      3      4 
	3      9      5 
	4      8      6
	5      2      7
};
\end{axis}
\end{tikzpicture}

\begin{tikzpicture}
    \begin{axis}[
    axis x line = center,
    axis y line = center,
    minor x tick num=1,
    xlabel={$x$},
    xmin=-10,xmax=10,
    ylabel={$y$},
    ymin=-10,ymax=10,  
    ]
    \addplot[magenta] {x^3};
    \end{axis}
\end{tikzpicture}
\vspace{2cm}
\begin{tikzpicture}
\begin{axis} [
    legend pos = north west, 
    ymin = 0, 
    grid = major
]
\legend{ 
	$\log_2(x)$, 
	$\ln(x)$, 
	$\log_{10}(x)$
};
\addplot {log2(x)};
\addplot {ln(x)};
\addplot {log10(x)};
\end{axis}
\end{tikzpicture}
\vspace{2cm}

\begin{tikzpicture}
\begin{axis}
\addplot3[
    surf,
] 
coordinates {
(0,0,0) (0,1,0) (0,2,0)

(1,0,0) (1,1,0.6) (1,2,0.7)

(2,0,0) (2,1,0.7) (2,2,1.8)
};
\end{axis}
\end{tikzpicture}

Как можно заметить, из последнего примера пакет textbf{tikz} позволяет выполнять построение 3D объектов. Однако, мы не стали затрагивать его слишком подробно, ибо не обладаем столь глубокими познаниями в геометрии. 
\end{document}