\documentclass[a4paper,12pt]{report}
%\documentclass[14pt, a4paper]{extarticle}

\usepackage[utf8]{inputenc}
\usepackage[russian]{babel}
\usepackage[left=2cm,right=2cm,
    top=1cm,bottom=2cm,bindingoffset=0cm]{geometry}
    %\usepackage[dvips]{graphicx}
    \usepackage{graphicx}%пакет для вставки изображений
    %\graphicspath{{piсs/}}%указываем откуда следует вставлять изображения
\DeclareGraphicsExtensions{.png,.jpg,.svg}%какое расширение будем использовать 
%\usepackage{fullpage}
\usepackage{longtable}
\usepackage{subcaption}

\begin{document}

%\newlength{\mytextsize} % определяем высоту шрифта
%\makeatletter
%\setlength{\mytextsize}{\f@size pt}
%\makeatother
\author{Буданцев Артем}

\begin{titlepage} %Титульник

\begin{center}
{\large МИНИСТЕРСТВО НАУКИ И ВЫСШЕГО ОБРАЗОВАНИЯ РОССИЙСКОЙ ФЕДЕРАЦИИ}\\

{\small Федеральное государственное бюджетное образовательное учреждение высшего образования
«Кемеровский государственный университет»
Институт фундаментальных наук
Кафедра ЮНЕСКО по информационным вычислительным технологиям
} \\
\vspace{2cm}
{\LARGE \textbf{Отчет}}\\
по учебной практике, технологической (проектно-технологической) практике
\vspace{0.5cm}


{\large проект ``Инструменты для оформления научных статей и презентаций на примере \LaTeX'}

\vspace{4cm}

\begin{flushleft}
\textbf{Выполнили:}
\end{flushleft}
студенты направления подготовки 02.03.03 Математическое обеспечение и администрирование информационных систем 
\end{center}

\begin{flushright}
\begin{tabular}{lp{1pt}l} 
    Басалаев Дмитрий && \hspace{2cm} \\\cline{1-1}\cline{3-3} 
     {\tiny Ф.И.О.}     && {\tiny Оценка}
  \end{tabular}
  
  \begin{tabular}{lp{1pt}l} 
    Болковая Полина && \hspace{2cm} \\\cline{1-1}\cline{3-3} 
     {\tiny Ф.И.О.}     && {\tiny Оценка}
  \end{tabular}
  
   \begin{tabular}{lp{1pt}l} 
    Буданцев Артём && \hspace{2cm} \\\cline{1-1}\cline{3-3} 
      {\tiny Ф.И.О.}     && {\tiny Оценка} 
  \end{tabular}
\end{flushright}

\vspace{6cm}
\begin{center}
{\Large Кемерово 2021}
\end{center}
\end{titlepage}

\newpage%Содержание
\tableofcontents


\newpage%Првая глава
\section{``Описание проекта'}
Краткое описание: Составить презентацию и отчет о проделанной работе при помощи \LaTeX, задействовав как можно больше его возможностей.Возможно подготовить небольшую справку по интерфейсу \TeX maker.
\subsection{Актуальность, теоретическая и практическая значимость}
\textbf{Актуальность:}
Издательский пакет LateX позволяет качественно оформить любой документ или презентацию, не задумываясь о её внешнем виде, а лишь сосредоточившись на изложении и структуре. С его помощью можно легко подготовить любой документ, начиная от доклада или объемного конспекта до семестровой или курсовой работы с многочисленными формулами.
\subsection{Теоретическая значимость}
\begin{itemize}
  \item Знакомоство студентов с издательским пакетом \LaTeX, описание его примуществ и недостатков
  \item Обзор интерфейса наиболее популярного \TeX редактора ``\TeX maker'.
 \item Получение нами умения создать качественные pdf документов 
\end{itemize} 
\subsection{Состав группы участников проекта}
\subsection{Состав группы}
\begin{tabular}{| l| l| l| l|}
\hline {\bfseries \large №} & {\bfseries \large ФИО} & {\bfseries \large \textsl{группа}} & {\bfseries \large Логин на github.com } \\ \hline
1. & Басалаев Д.А.  & МОА-205 & FySyZe \\ \hline
2. & Болковая П.А.  & МОА-205 & ApollinariaB \\ \hline
3. & Буданцев А.А.  & МОА-205 & Antur1um \\ \hline
\end{tabular}
\subsection{Общие цель и задачи}
\textbf{Цель:} Составить презентацию и отчет о проделанной работе при помощи LateX, задействовав как можно больше его возможностей.Возможно подготовить небольшую справку по интерфейсу Texmaker.
\subsection{Распределение по ролям}
\textbf{Басалаев Д.А.} Работа с презнтациями, форматирование страницы\\
\textbf{Болковая П.А.} Работа с изображениями и встроенной графикой\\
\textbf{Буданцев А.А.} Ввод формул, построение графиков, различные окружения
\subsection{План-график работы}
\begin{tabular}{| l| p{13cm}|}
\hline {\bfseries \large Даты} & {\bfseries \large Действия}\\ \hline
03.02.21-11.03.21. & Изучение базы, установка необходимого софта,подготовка документации\\ \hline
12.03.21-26.03.21 & Изучение интерфейса в \TeX maker, набор простых текстов, спецсимволы \\ \hline
27.03.21-15.04.21 & Ввод математических формул, ввод матриц, спецсимволы  \\ \hline
16.04.21-28.04.21 & Работа с изображениями и встроенной графикой, построение графиков \\ \hline 
29.04.21-14.05.21 & Работа с ссылками,разметка страницы, различные окружения, работа с графикой и презентациями \\ \hline
15.05.21- & Разработка финального продукта, подготовка отчета. \\ \hline
\end{tabular}
\subsection{Что такое \TeX и \LaTeX ?}
\textbf{\TeX} — издательская система, созданная американским математиком и программистом Дональдом Кнутом (Donald E. Knuth). TEX был разработан, преследуя две основные цели: - позволить всем создавать качественные публикации с разумными для этого усилиями. \TeX знаменит своей чрезвычайной стабильностью, работой на различных операционных системах и практически полным отсутствием ошибок. Одна из главных причин по которой \TeX выбирают для оформления научных работ заключается в том, что с его помощью можно достаточно легко вводить сложные формулы.\\

\textbf{\LaTeX} — наиболее популярный набор макрорасширений (или макропакет) системы компьютерной вёрстки \TeX, который облегчает набор сложных документов.Первая версия \LaTeX была написана в 1984 году Лесли Лампортом (Leslie Lamport) и с тех пор стала доминирующим способом подготовки \TeX публикаций. Важно заметить, что ни один из макропакетов для \TeX ’а не может расширить \TeX ’овских возможностей (всё, что можно сделать в LaTeX’е, можно сделать и в \TeX ’е), но, благодаря различным упрощениям, использование макропакетов зачастую позволяет избежать весьма изощрённого программирования.Пакет позволяет автоматизировать многие задачи набора текста и подготовки статей, включая набор текста на нескольких языках, нумерацию разделов и формул, перекрёстные ссылки, размещение иллюстраций и таблиц на странице, ведение библиографии и др. Кроме базового набора существует множество пакетов расширения \LaTeX.

\subsection{Используемые программные средства}
\begin{itemize}
\item[1.]Github

\item[2.]\TeX Live

\item[3.]\TeX maker
\end{itemize}
Для того чтобы использовать \LaTeX на современном ПК под управлением Windows 10 нам понадобится загрузить и установить \TeX live maneger(это наиболее полный дистрибутив \LaTeX), а также \TeX maker(это редактор для создания TEX документов). А для сохранения документов в формате pdf нам понадобится написать пару строк в командной строке.

\subsection{Что представляет собой \LaTeX докумет}
\LaTeX документ состоит из двух частей: файл с расширением .tex в котором содержатся обычный текст и команды \LaTeX(входной файл) и собственно скомпилированный pdf файл(выходной файл). Для того чтобы получить pdf файл из .tex файла нам необходимо зайти в командную строку, затем при помощи команды "cd" перейти в директорию в которой лежит .tex файл затем написать команду "pdflatex" и название файла с указанием расширения (.tex).(например: pdflatex FinalReport.tex)

\section{Ход работы}

\subsection{03.02.21-11.03.21}
Загрузили \TeX live maneger и \TeX maker. Ознакомились с интерфейсом, синтаксисом набора команд и структурой документа. Подготовили документацию по проекту.

\subsection{12.03.21-26.03.21}
Изучили набор команд для написания спец. символов и изменения шрифта(\{ \textbf{жирный}, \textsl{Курсив}, {\tiny крошечный} {\Huge Огромный} \} \$ \texteuro \   и др.)
Решили составить таблицу, содержащую наиболее часто используемые команды, но вскоре отказались от этой идеи ибо в \TeX maker присутствуют автоматические подсказки, а также многие действия вынеcены на кнопки интерфейса. 




\begin{figure}[h]
\begin{tabular}{cc}
\includegraphics[width=9cm,height=7cm]{table(1)}
&
\includegraphics[width=9cm,height=7cm]{table(2)}
\end{tabular}
\caption{Та самая недоделанная таблица}
\end{figure}
\newpage
\begin{figure}[h!]
\setlength{\fboxsep}{0pt}%
\setlength{\fboxrule}{1pt}%ширина рамки
\fbox{\includegraphics[width=15cm,height=9cm]{TableCode}}%
\caption{Код таблицы}
\label{fig:image}
\end{figure}




\subsection{27.03.21-15.04.21}
Итак, мы приступили к вводу математических выражений и формул. Желая начать с чего-то простого мы решили преписать школьную таблицу производных и интегралов.
\begin{figure}[h!]
\setlength{\fboxsep}{0pt}%
\setlength{\fboxrule}{1pt}%ширина рамки
\fbox{\includegraphics[width=15cm,height=9cm]{Example(1)}}%
\caption{Уже на этом этапе можно понять наскольно в \LaTeX \  проще и быстрее вводить математематические формулы}
\label{fig:image}
\end{figure}
\newpage
Итак, быстро убедившись что ввод сложных математических формул не представляет трудностей мы приступили к вводу матриц и других крупных объектов.
\begin{figure}[h!]
\setlength{\fboxsep}{0pt}%
\setlength{\fboxrule}{1pt}%ширина рамки
\fbox{\includegraphics[width=15cm,height=9cm]{Matrrix(1)}}%
\caption{"Крупные' математические объекты}
\end{figure}
\\
Что же касается спец символов, в \LaTeX 'e их огромное количество,(к счастью) но раз уж речь идет о математике, то давайте попробуем собрать определение последовательности на языке " $\varepsilon \ \Delta $'
$$\lim_{x \rightarrow x_0}f(x) = A \  \Leftrightarrow \   \forall \ \varepsilon > 0, \ \exists \delta \ >0, |\forall x \ 0<|x-x_0|<\delta \ \Rightarrow |f(x) - A|< \varepsilon   $$


$$\lim_{x \rightarrow 0}\frac{\sin x}{x} = 1  $$

$$\lim_{x \rightarrow \infty} \left( 1 + \frac{1}{x} \right)^x = e  $$

%$\lim_{x \rightarrow 0} \dfrac{sinx}{x} = 1$

\subsection{16.04.21-28.04.21}
Для работы с графикой мы решили рассмотреть два пакета: \{graphicx\} и \{tikz\} первый служит для вставки растровых изображений в текст,  а второй позволяет выполнять построение различных геометрических фигур, блок-схем, а тыкже графиков некоторых функций, что представляет гораздо больший интерес. Начнем с пакета \{graphicx\}. \\
Для нам нужно подключить его в преамбуле документа:\\
\\
$\backslash$\textbf{usepackage \{graphicx \}} \\
\textbf{\graphicspath\{\{pictures/ \}\} } Указываем название каталога где будут лежать изображения.(Он должен находиться в том же катологе что и сам документ) Данная опция является необязательной, можно просто рассметить все изображения в том же каталоге что и документ.\\
$\backslash$ \textbf{DeclareGraphicsExtensions\{.pdf,.png,.jpg\}} Указываем какие типы файлов будем использовать.Векторные изображения также поддерживаются. \\
Рассмотрим вставку изображений:\\
$\backslash$ \textbf{begin \{figure\}[h!]} "Обьявляем начало" изображения, в квадратных скобках указываем позицию изображения, "h!" обозначает, что изображение будет вставлено сразу после текста.       \\
$\backslash$ \textbf{ setlength \{ $\backslash$ fboxsep \} \{0pt \} } размер полей вокруг изображения     \\
$\backslash$ \textbf{setlength \{ $\backslash$ fboxrule \} \{ 1pt \} } ширина рамки \\
$\backslash$ \textbf{fbox \{ $\backslash$ includegraphics [ width=15cm,height=9cm ]\{ Matrrix \( 1 \) \} \} } задаем размеры изображения и указываем название файла(файл должен лежать в одной папке с документом)    \\%
$\backslash$ \textbf{caption \{ "Крупные' математические объекты \}}| Подпись под изображением  \\
$\backslash$ \textbf{end \{ figure \}  } "Конец" \\
Если подпись и рамка не трубуются, то достачно только строчки $\backslash$ \textbf{fbox \{ $\backslash$ includegraphics [ width=15cm,height=9cm ]\{ Matrrix \( 1 \) \} \} }. Вместо усазания размеров в сантиметрах можно использовать команду \textbf{scale} \( масштаб\). \\
Итак, перейдем к пакету \textbf{\{tikz\}}.

















\end{document}
